\input{slides_template}	% nothing to do here
\input{c_introduction_info} % TODO modify this if you have not already done so

\usepackage{tabularx}

% meta-information
\newcommand{\topic}{
    Variables and user input
}

% nothing to do here
\title{\topic}
\supertitle{\course}
\date{}

% the actual document
\begin{document}

\maketitle

\begin{frame}{Contents}
    \tableofcontents
\end{frame}

\section{Variables}
\subsection{}

\begin{frame}[fragile]{Usage}
    Declaration:
    \begin{lstlisting}[numbers=none,basicstyle=\itshape\footnotesize]
type identifier;
\end{lstlisting}
    Assignment: 
    \begin{lstlisting}[numbers=none,basicstyle=\itshape\footnotesize]
identifier = value;
\end{lstlisting}
    Definition (all at once):
    \begin{lstlisting}[numbers=none,basicstyle=\itshape\footnotesize]
type identifier = value;
\end{lstlisting}
    Example:
    \begin{lstlisting}[numbers=none]
unsigned number;                // declaration (*avoid this)
number = 42;                    // assignment
unsigned another_number = 23;   // definition

\end{lstlisting}
\end{frame}

\begin{frame}[fragile]{Valid identifiers}
    \begin{itemize}
        \item Consist of English letters (no \textit{ß}, \textit{ä}, \textit{ö}, \textit{ü}), numbers and underscore (\_)
        \item Start with a letter or underscore
        \item Are case sensitive (\textit{number} differs from \textit{Number})
        \item should not begin with double underscore or underscore followed by uppercase (reserved)
        \item Must not be reserved words:
    \end{itemize}
    {\scriptsize\tt\begin{tabular}{ccccccccc}
alignas     & char      & double    & for       & nullptr   & sizeof            & true          & void \\
alignof     & const     & else      & goto      & register  & static            & typedef       & volatile \\
auto        & constexpr & enum      & if        & restrict  & static\_assert    & typeof        & while \\
bool        & continue  & extern    & inline    & return    & struct            & typeof\_unqual & \\
break       & default   & false     & int       & short     & switch            & union         & \\
case        & do        & float     & long      & signed    & thread\_local     & unsigned      &  
    \end{tabular}}
    \\\bigskip
    \textbf{Style}:\\
    \begin{itemize}
        \item Stay in one language (i.e. stay in English)
        \item Decide whether to use \textit{camelCaseIdentifiers} or \textit{snake\_case\_identifiers}.\\
    \end{itemize}
\end{frame}

\begin{frame}[fragile]{Use modern C style!}
    Oftentimes when you read old C code, it will look like this (because it was necessary in C89):
    \begin{lstlisting}
unsigned a, b, c;
a = 3;
b = 2;
c = (1 / 3) * a * a * b;
\end{lstlisting}

    However if you forget to assign a variable you use, bad things will happen.\\
    \bigskip
    Instead always define variables you declare (or initialize them to some default):
    \begin{lstlisting}
unsigned a = 3;
unsigned b = 2;
unsigned c = (1 / 3) * a * a * b;
\end{lstlisting}
\end{frame}

\begin{frame}[fragile]{Use descriptive identifiers!}
    Readers of your code (future you) will wonder what a, b and c mean...
    \begin{lstlisting}
// calculate the volume of a pyramid
unsigned a = 3;
unsigned b = 2;
unsigned c = (1 / 3) * a * a * b;
\end{lstlisting}
    \centering$\Downarrow$\vspace*{5pt}
    \begin{lstlisting}
// calculate volume of square pyramid
unsigned length = 3;
unsigned height = 2;
unsigned volume = (1 / 3) * length * length * height;
\end{lstlisting}
\end{frame}

\begin{frame}{Use descriptive identifiers.}
    \LARGE
    \centering
    Please, use descriptive identifiers.\footnotemark
    
    \footnotetext[1]{Seriously, use descriptive identifiers.}
\end{frame}

\section{Data types}
\subsection{}

\begin{frame}[fragile]{Basic types in C}
    In C, there are 3 kinds of basic types:
    \begin{itemize}
        \item the Character type: \lstinline{char}
        \item Integer types (signed and unsigned)
        \item Floating-Point types
    \end{itemize}
    \bigskip
    Let's look at those types in more detail
\end{frame}

\begin{frame}[fragile]{Characters}
    \begin{itemize}
        \item Keyword: \lstinline{char}
        \item Size: always 1 Byte (can in theory be more than 8 Bits)
        \item Intended to represent a single character
        \item Stores its \textit{ASCII} number (e.g. 'A' $\Rightarrow$ 65)
    \end{itemize}\ \\
    \ \\
    You can define a \textit{char} either by its ASCII number or by its symbol:
    \begin{lstlisting}[numbers=none]
char a_dec = 65;       // ASCII code for A
char a_hex = 0x41;     // hexadecimal ASCII code for A
char a_chr = 'A';	   // use single quotation marks
\end{lstlisting}
\end{frame}

\begin{frame}[fragile]{Signed integer types}
    \begin{itemize}
        \item Possible keywords: $\lstinline{short} \le \lstinline{int} \le \lstinline{long} \le \lstinline{long long}$
        \item You can optionally prefix them with \lstinline{signed}
        \item Actual size of \lstinline{int}, \lstinline{short}, \lstinline{long} depends on architecture
        \item If these types overflow, bad things may happen
        \item \textbf{if you do signed arithmetic, \lstinline{int} is a good default}
    \end{itemize}\ \\
    \ \\
    Example:
    \begin{lstlisting}[numbers=none]
int a;              // Range (on x86/64): -2.147.483.648 to 2.147.483.647
signed int a;       // Same es before (explicit signed)
signed a;           // Same as before (int can be omitted)

short b;            // Range (on x86/64): -32.768 to 32.767
signed short b;     // Same as before
\end{lstlisting}
\end{frame}

\begin{frame}[fragile]{Unsigned integer types}
    \begin{itemize}
        \item All signed integer types can also be \lstinline{unsigned}
        \item \lstinline{unsigned char} is used to represent raw bytes
        \item \lstinline{bool} is used to represent \lstinline{true} {\small(1)} or \lstinline{false} {\small(0)} \hspace*{1em}(requires \lstinline{#include <bool.h>})
        \item if unsigned types overflow, they just wrap around to 0
    \end{itemize}\ \\
    \ \\
    Example:
    \begin{lstlisting}[numbers=none]
#include <bool.h>   // allow the usage of the bool type

unsigned int a;     // Range (on x86/64): 0 to 4.294.967.296
unsigned a;         // Same as before (int can be omitted)

unsigned char b;    // Range (one Byte): 0 to 255
bool c;             // Can be true (=1) or false (=0)
\end{lstlisting}
\end{frame}

\begin{frame}[fragile]{Floating point numbers}
    \begin{itemize}
        \item Keywords: \lstinline{float}, \lstinline{double}, \lstinline{long double}
        \item Stored as specified in \textit{IEEE 754 Standard} TL;DR
        \item Special values for $\infty$, $-\infty$, NaN
        \item Useful for fractions and very large numbers
        \item Type a decimal point instead of a comma!
        \item \textbf{usually you should use \lstinline{double} for floating point operations}
    \end{itemize}\ \\
    \ \\
    Example:
    \begin{lstlisting}[numbers=none]
float x = 0.125;            // Precision: 7 to 8 digits
double y = 111111.111111;   // Precision: 15 to 16 digits
\end{lstlisting}
\end{frame}

\begin{frame}{Basic types summary}
    \centering
    \renewcommand{\arraystretch}{1.5}
    \begin{tabularx}{0.9\linewidth}{|l|X|X|}
        \hline
        \textbf{type} & \textbf{use case} & \textbf{typical values}\\
        \hline
        \lstinline{bool} & truth values & \lstinline{true}, \lstinline{false}, 1, 0\\
        \lstinline{char} & character handling & \lstinline{'A'} .. \lstinline{'Z'}, 0 .. 127\\
        \lstinline{unsigned} & counting, arithmetic & 0 .. 4,294,967,295 \\
        \lstinline{int} & signed arithmetic, error handling\footnote{not great, but used in many libraries} & -2,147,483,648 .. 2,147,483,647 \\
        \lstinline{double} & floating point arithmetic & 3.14159 \\
        \hline
    \end{tabularx} 
\end{frame}

\section{Variable I/O}
\subsection{}
\begin{frame}[fragile]{\textit{printf()} with placeholders}
    The string you pass to \textit{printf} may contain placeholders:
    \begin{lstlisting}[numbers=none]
int a = -3;
unsigned b = 5;
double c = 7.4;

printf("a: %d\n", a);
printf("b: %u\nc: %f\n", b, c);
\end{lstlisting}
Output:\begin{lstlisting}[numbers=none]
a: 3
b: 5
c: 7.4
\end{lstlisting}
You can insert any amount of placeholders. For each placeholder, you have to pass a value of the corresponding type.
\end{frame}
\begin{frame}{Example placeholders}
    The placeholder determines how the value is interpreted.
    To avoid compiler warnings, only use the following combinations: \\ \ \\
    \centering
    \renewcommand{\arraystretch}{1.5}
    \begin{tabular}{|c|c|c|}
        \hline
        \textbf{type} & \textbf{description} & \textbf{type of argument} \\\hline
        \%c & single character & char, int (if $<=$ 255) \\\hline
        \%d & decimal number & char, int \\\hline
        \%u & unsigned decimal number & unsigned char, unsigned int \\\hline
        \%X & hexadecimal number & char, int \\\hline
        \%ld & long decimal number & long \\\hline
        \%f & floating point number & float, double \\\hline
    \end{tabular}
\end{frame}
\begin{frame}[fragile]{Variable input}
    \textit{scanf()} is another useful function from the standard library.
    \begin{itemize}
        \item Like \textit{printf()}, it is declared in \textit{stdio.h}
        \item Like \textit{printf()}, it has a format string with placeholders
        \item You can use it to read values of primitive datatypes from the command line
    \end{itemize}
    \ \\ \ \\ Example:
    \begin{lstlisting}[numbers=none]
int i = 0;
scanf("%d", &i);	
\end{lstlisting}
    After calling \textit{scanf()}, the program waits for the user to input a value in the command line.
    After pressing the \textit{return} key, that value is stored in \textit{i}.
\end{frame}
\begin{frame}{Note:}
    \begin{itemize}
        \item \textit{scanf()} uses the same placeholders as \textit{printf()}
        \item You must type an \textit{\&} before each variable identifier \\
            (more about this later)
        \item If you read a number (using \%d, \%u etc.), interpretation
        \begin{itemize}
            \item Starts at first digit
            \item Ends before last non digit character
        \end{itemize}
        \item If you use \%c, the first character of the user input is interpreted\\
        (this may be a \lstinline{'\\n'} or \lstinline{'\\0'} as well!)
    \end{itemize}
    Never trust the user: they may enter a blank line while you expect a number, which means your input variable is still undefined!\\
    Always initialize your variables!
\end{frame}

% nothing to do from here on
\end{document}
